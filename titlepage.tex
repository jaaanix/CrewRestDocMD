\begin{titlepage}
\newcommand{\HRule}{\rule{\linewidth}{0.5mm}}
\center
\textsc{\LARGE Technische Hochschule Köln} \\[1.5cm]
\textsc{\Large Allgemeine Informatik} \\[0.5cm]
\textsc{\large Praxisprojekt} \\
\HRule  \\[0.4cm]
{ \huge \bfseries Xamarin Cross-Plattform App CrewRest} \\[0.4cm]
\HRule  \\[1.5cm]
\vfill
\begin{minipage}[t]{0.4\textwidth}
    \begin{flushleft} \large
        \emph{Autor:} \newline
        \textsc{Janis Saritzoglou}
        \newline
        \emph{Matrikel Nr.:}
        \textsc{11091991}
    \end{flushleft}
\end{minipage}
\hfill
\begin{minipage}[t]{0.5\textwidth}
    \begin{flushright} \large
        \begin{flushleft}
            \emph{Betreuer:} \newline
            \textsc{Prof. Dr. Christian Kohls}
    \end{flushleft}
    \end{flushright}
\end{minipage}
\vfill
{\large \today}
\end{titlepage}
% Abstract
\begin{center} \textbf{\uppercase{abstract}} \end{center}
Im Folgenden Praxisprojekt ist es Ziel die Cross-Plattform Applikation "CrewRest" für die Betriebssysteme Windows 10, Android und iOS zu erstellen. Die zu erstellende Applikation soll als Prototyp zur Demonstration bei einer oder mehrerer Luftfahrtgesellschaften dienen. Wichtig ist dabei eine Umsetzung für unterschiedliche Plattformen mit möglichst geringem Aufwand. Als Technologien sollen Microsoft UWP und Xamarin zum Einsatz kommen. Somit soll aus einer gemeinsamen Codebasis eine Applikation, lauffähig auf verschiedenen Betriebssystemen erzeugt werden. Als Programmiersprache kommt C\# zum Einsatz. Der fachliche Hintergrund ist es, ein System für die Mitarbeiter einer Luftfahrtgesellschaft zu entwickeln, um Anträge für Abwesenheiten wie z.B. Urlaubsanträge erstellen, sowie dessen Status einsehen zu können. Bei Möglichkeit ist es wünschenswert eine Art Kalenderüberischt in der Applikation anzuzeigen. Diese Übersicht, welche die Abwesenheitslage in bestimmten Zeiträumen aller Mitarbeiter darstellt, soll dem Benutzer ermöglichen sinnvolle Zeiträume für neue Abwesenheitsanträge erstellen zu können.
